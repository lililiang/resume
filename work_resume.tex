%% start of file `template-zh.tex'.
%% Copyright 2006-2013 Xavier Danaux (xdanaux@gmail.com).
%
% This work may be distributed and/or modified under the
% conditions of the LaTeX Project Public License version 1.3c,
% available at http://www.latex-project.org/lppl/.


\documentclass[11pt,a4paper,sans]{moderncv}   % possible options include font size ('10pt', '11pt' and '12pt'), paper size ('a4paper', 'letterpaper', 'a5paper', 'legalpaper', 'executivepaper' and 'landscape') and font family ('sans' and 'roman')

% moderncv 主题
\moderncvstyle{casual}                        % 选项参数是 ‘casual’, ‘classic’, ‘oldstyle’ 和 ’banking’
\moderncvcolor{blue}                          % 选项参数是 ‘blue’ (默认)、‘orange’、‘green’、‘red’、‘purple’ 和 ‘grey’
%\nopagenumbers{}                             % 消除注释以取消自动页码生成功能

% 字符编码
\usepackage[utf8]{inputenc}                   % 替换你正在使用的编码
\usepackage{CJKutf8}

% 调整页面出血
\usepackage[scale=0.75]{geometry}
%\setlength{\hintscolumnwidth}{3cm}           % 如果你希望改变日期栏的宽度

% 个人信息
\name{}{李亮}
\title{PHP开发工程师}                     % 可选项、如不需要可删除本行
%\address{街道及门牌号}{邮编及城市}            % 可选项、如不需要可删除本行
\phone[mobile]{130~2714~6059}              % 可选项、如不需要可删除本行
%\phone[fixed]{+2~(345)~678~901}               % 可选项、如不需要可删除本行
%\phone[fax]{+3~(456)~789~012}                 % 可选项、如不需要可删除本行
\email{liliang\_hust@163.com}                    % 可选项、如不需要可删除本行
\homepage{blog.csdn.net/asdfglili}                  % 可选项、如不需要可删除本行
\extrainfo{Powered by TeX}                 % 可选项、如不需要可删除本行
\photo[64pt][0.4pt]{resume}                  % ‘64pt’是图片必须压缩至的高度、‘0.4pt‘是图片边框的宽度 (如不需要可调节至0pt)、’picture‘ 是图片文件的名字;可选项、如不需要可删除本行
%\quote{来源自V2EX}                          % 可选项、如不需要可删除本行

% 显示索引号;仅用于在简历中使用了引言
%\makeatletter
%\renewcommand*{\bibliographyitemlabel}{\@biblabel{\arabic{enumiv}}}
%\makeatother

% 分类索引
%\usepackage{multibib}
%\newcites{book,misc}{{Books},{Others}}
%----------------------------------------------------------------------------------
%            内容
%----------------------------------------------------------------------------------
\begin{document}
\begin{CJK}{UTF8}{gbsn}                       % 详情参阅CJK文件包
\maketitle

\section{教育背景}
\cventry{2011年 -- 2015年}{本科}{华中科技大学}{985}{\textit{TOP 25\%}}{}  % 第3到第6编码可留白
%\cventry{年 -- 年}{学位}{院校}{城市}{\textit{成绩}}{说明}

%\section{毕业论文}
%\cvitem{题目}{\emph{题目}}
%\cvitem{导师}{导师}
%\cvitem{说明}{\small 论文简介}

\section{工作背景}
\subsection{全职\&实习}
\cventry{2014年10月 -- 2016年10月}{PHP开发工程师}{杭州王耀网络科技有限公司}{杭州}{}{主要负责公司web端以及移动端后台的开发迭代和维护\newline{}%
工作内容:%
\begin{itemize}%
\item 负责web端产品的开发、优化、维护:
  \begin{itemize}
  \item 面对用户的快速上涨(数十万~数百万),参与进行架构的重构,通过引入缓存系统,异步任务系统,拆分RPC服务等方法,很好的完成了从小型网站到中型网站规模的架构演进,以应对数百万级别用户的服务
    \begin{itemize}
    \item 引入以Redis缓存结构,利用Redis多样的数据结构来支撑后台逻辑业务;
    \item 利用开源异步任务框架,来搭建自己的异步任务系统,支撑某些业务需求;
    \item 利用高性能Swoole扩展,拆分第三方服务,以应对频次较高的用户服务请求;
    \end{itemize}
  \item 面对日常业务需求,利用TP框架完成日常开发任务,主要负责后台逻辑的开发,以及维护15W+行核心代码,快速优质的满足了日常的开发需求
  \end{itemize}
\item 负责移动端后台接口的开发以、优化、维护:
  \begin{itemize}
  \item 面对公司启动移动端的产品策略,负责移动端接口的整体项目,通过修改TP框架内核以适应移动端接口的开发需求,独立开发维护全部400+接口,引入开源文档管理工具来维护接口文档,后续修改框架来实现更优雅的代码组织方式,以保证接口服务的健壮可用
  \end{itemize}
\item 负责用户时间轴系统的开发、优化、维护:
  \begin{itemize}
  \item 为了满足用户个性化的产品体验,重新构建了Timeline系统,通过高性能的Swoole扩展,以及自身的推荐算法机制,实现了足以支撑百万级用户量的个性化feed流推荐,提高了用户社区的活跃度,基本满足了向用户展示优质的个性化内容的要求。
  \end{itemize}
\item 主业务相关的子系统的开发维护、部署上线,包括后台管理系统,小游戏系统,消息推送系统等;
  \begin{itemize}
    \item 利用TP5 + bootstrap,搭建后台管理系统,以满足社区管理的业务需求;
    \item 利用TP5 + PHP Swoole 搭建全站的消息推送系统;
  \end{itemize}
\end{itemize}}
\subsection{实习}
\cventry{2014年07月 -- 2014年09月}{软件开发实习生}{上海自仪泰雷兹交通自动化公司}{上海}{}{负责维护公司开发文档,负责参与日常开发会议,了解项目流程,完善开发文档}


\section{计算机技能}
\cvdoubleitem{PHP}{TP, Lavarel}{++++}{}
\cvdoubleitem{NOSQL}{Redis}{+++}{}
\cvdoubleitem{SQL}{MySQL}{+++}{}
\cvdoubleitem{C}{C98}{+++}{}
\cvdoubleitem{服务器}{Apache, Nginx}{++}{}
\cvdoubleitem{OS}{Ubuntu, Debian, CentOS}{++}{}
\cvdoubleitem{Python}{flask, scrapy}{++}{}
\cvdoubleitem{前端}{HTML, CSS, JS}{++}{}
\cvdoubleitem{算法}{k-means, CF等}{+}{}


\section{个人项目经历}
\cvlistitem{基于半监督学习的人脸识别系统}
\cvlistitem{基于BOW模型的模式识别}
\cvlistitem{基于CodeIgniter框架的管理系统}
\cvlistitem{基于百度地图的web动画}
\cvlistitem{物流仿真管理软件}

\section{语言技能}
\cvitemwithcomment{英语}{CET--4}{}
%\cvitemwithcomment{语言 2}{水平}{评价}

\section{个人兴趣}
\cvitem{读书}{\small 最近读一些计算机原理相关的书,如《深入理解计算机系统》等}
\cvitem{健身}{\small 保持身体健康}
\cvitem{看戏}{\small 喜欢传统文化}

% 来自BibTeX文件但不使用multibib包的出版物
%\renewcommand*{\bibliographyitemlabel}{\@biblabel{\arabic{enumiv}}}% BibTeX的数字标签
\nocite{*}
\bibliographystyle{plain}
\bibliography{publications}                    % 'publications' 是BibTeX文件的文件名

% 来自BibTeX文件并使用multibib包的出版物
%\section{出版物}
%\nocitebook{book1,book2}
%\bibliographystylebook{plain}
%\bibliographybook{publications}               % 'publications' 是BibTeX文件的文件名
%\nocitemisc{misc1,misc2,misc3}
%\bibliographystylemisc{plain}
%\bibliographymisc{publications}               % 'publications' 是BibTeX文件的文件名

\clearpage\end{CJK}
\end{document}


%% 文件结尾 `template-zh.tex'.
